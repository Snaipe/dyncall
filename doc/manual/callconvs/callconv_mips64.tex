%//////////////////////////////////////////////////////////////////////////////
%
% Copyright (c) 2007,2009 Daniel Adler <dadler@uni-goettingen.de>, 
%                         Tassilo Philipp <tphilipp@potion-studios.com>
%
% Permission to use, copy, modify, and distribute this software for any
% purpose with or without fee is hereby granted, provided that the above
% copyright notice and this permission notice appear in all copies.
%
% THE SOFTWARE IS PROVIDED "AS IS" AND THE AUTHOR DISCLAIMS ALL WARRANTIES
% WITH REGARD TO THIS SOFTWARE INCLUDING ALL IMPLIED WARRANTIES OF
% MERCHANTABILITY AND FITNESS. IN NO EVENT SHALL THE AUTHOR BE LIABLE FOR
% ANY SPECIAL, DIRECT, INDIRECT, OR CONSEQUENTIAL DAMAGES OR ANY DAMAGES
% WHATSOEVER RESULTING FROM LOSS OF USE, DATA OR PROFITS, WHETHER IN AN
% ACTION OF CONTRACT, NEGLIGENCE OR OTHER TORTIOUS ACTION, ARISING OUT OF
% OR IN CONNECTION WITH THE USE OR PERFORMANCE OF THIS SOFTWARE.
%
%//////////////////////////////////////////////////////////////////////////////

\subsection{MIPS64 Calling Convention}

\paragraph{Overview}

There are two main ABIs in use for MIPS64 chips, \emph{N64}\cite{MIPSn32/n64} and \emph{N32}\cite{MIPSn32/n64}. Both are
basically the same, except that N32 uses 32-bit pointers and long integers, instead of 64. All registers of a MIPS64 chip are considered
to be 64-bit wide, even for the N32 calling convention.\\
The word size is defined to be 32 bits, a dword 64 bits. Note that this is due to historical reasons (terminology didn't change from MIPS32).\\
Other than that there are 64-bit versions of the other ABIs found for MIPS32, e.g. the EABI\cite{MIPSeabi} and O64\cite{MIPSo64}.

\paragraph{\product{dyncall} support}

For MIPS 64-bit machines, dyncall supports the N32 and N64 calling conventions for calls and callbacks.
Our test machine is a Loongson-CPU 2F subnotebook, as well as an EdgeRouter Lite, both running OpenBSD.

\subsubsection{MIPS N64 Calling Convention}

\paragraph{Register usage}

\begin{table}[h]
\begin{tabular*}{0.95\textwidth}{lll}
Name                                   & Alias                & Brief description\\
\hline
{\bf \$0}                              & {\bf \$zero}         & Hardware zero \\
{\bf \$1}                              & {\bf \$at}           & Assembler temporary \\
{\bf \$2-\$3}                          & {\bf \$v0-\$v1}      & Integer results \\
{\bf \$4-\$11}                         & {\bf \$a0-\$a7}      & Integer arguments, or double precision float arguments\\
{\bf \$12-\$15,\$24}                   & {\bf \$t4-\$t7,\$t8} & Integer temporaries \\
{\bf \$25}                             & {\bf \$t9}           & Integer temporary, hold the address of the called function for all PIC calls (by convention) \\
{\bf \$16-\$23}                        & {\bf \$s0-\$s7}      & Preserved \\
{\bf \$26,\$27}                        & {\bf \$kt0,\$kt1}    & Reserved for kernel \\
{\bf \$28}                             & {\bf \$gp}           & Global pointer, preserve \\
{\bf \$29}                             & {\bf \$sp}           & Stack pointer, preserve \\
{\bf \$30}                             & {\bf \$s8}           & Frame pointer, preserve \\
{\bf \$31}                             & {\bf \$ra}           & Return address, preserve \\
{\bf hi, lo}                           &                      & Multiply/divide special registers \\
{\bf \$f0,\$f2}                        &                      & Float results \\
{\bf \$f1,\$f3,\$f4-\$f11,\$f20-\$f23} &                      & Float temporaries \\
{\bf \$f12-\$f19}                      &                      & Float arguments \\
{\bf \$f24-\$f31}                      &                      & Preserved \\%@@@on N32, this changes
\end{tabular*}
\caption{Register usage on MIPS N64 calling convention}
\end{table}

\paragraph{Parameter passing}

\begin{itemize}
\item Stack grows down
\item Stack parameter order: right-to-left
\item Caller cleans up the stack
\item first 8 integers (<= 64-bit)are passed in registers \$a0-\$a7
\item first 8 floating point arguments (single or double precision) are passed in registers \$f12-\$f19
\item if either integer or float registers are used up, the stack is used
\item all stack entries are 64-bit aligned
\item results are returned in \$v0, and for a second one \$v1 is used
\item float arguments passed in the variable part of a vararg call are passed like integers
\end{itemize}

\paragraph{Stack layout}

Stack directly after function prolog:\\

\begin{figure}[h]
\begin{tabular}{5|3|1 1}
\hhline{~-~~}
                                & \vdots       &                                &                              \\
\hhline{~=~~}                            
register save area              & \hspace{4cm} &                                & \mrrbrace{5}{caller's frame} \\
\hhline{~-~~}                            
local data                      &              &                                &                              \\
\hhline{~-~~}                            
\mrlbrace{3}{parameter area}    & \ldots       & \mrrbrace{3}{stack parameters} &                              \\
                                & \ldots       &                                &                              \\
                                & \ldots       &                                &                              \\
\hhline{~=~~}
register save area              & padding? @@@ &                                & \mrrbrace{7}{current frame}  \\
                                & \$ra         &                                &                              \\
                                & \$s8         &                                &                              \\
                                & \$gp         &                                &                              \\
\hhline{~-~~}
local data                      &              &                                &                              \\
\hhline{~-~~}
parameter area                  &              &                                &                              \\
\hhline{~-~~}
                                & \vdots       &                                &                              \\
\hhline{~-~~}
\end{tabular}
\caption{Stack layout on mips64 n64 calling convention}
\end{figure}

